

% \subsection{FWER-controlling procedures}
% \label{subsec:FWER-controlling-procedures}

We recall four thresholding procedures aimed at controlling family-wise error rates,
starting with the well-known Bonferroni's procedure.
\begin{definition}[Bonferroni's procedure]
Suppose the errors $\epsilon(j)$'s have a common marginal distribution $F$, Bonferroni's procedure with level $\alpha$ is the thresholding procedure that uses the threshold
\begin{equation} \label{eq:Bonferroni-procedure}
    t_p = F^{\leftarrow}(1 - \alpha/p).
\end{equation}
%where  $F^{\leftarrow}(u)=\inf{\left\{x:F(x)\ge u\right\}}$ is the generalized inverse function.
\end{definition}
% It is easy to see that the family-wise error rate (FWER) is controlled at level $\alpha$ by applying the union bound, regardless of the error-dependence structure (see e.g.\ Relation \eqref{eq:Bonferroni-FWER-control}, below).
The Bonferroni procedure is deterministic (i.e., non data-dependent), and only depends on the dimension of the problem and the null distribution.
A closely related procedure is Sid\'ak's procedure \citep{vsidak1967rectangular},
which is a more aggressive (and also deterministic) thresholding procedure that uses the 
threshold
\begin{equation} \label{eq:Sidak-procedure}
    t_p = F^{\leftarrow}((1 - \alpha)^{1/p}).
\end{equation}
% can be shown to control FWER in the case independent errors.

The third procedure, strictly more powerful than Bonferroni's, is the so-called Holm's procedure \citep{holm1979simple}.
On observing the data $x$, its coordinates can be ordered from largest to smallest
$x(i_1) \ge x(i_2)  \ge \ldots \ge x(i_p)$,
where $(i_1, \ldots, i_p)$ is a permutation of $\{1, \ldots, p\}$. 
Denote the order statistics as $x_{[1]}, x_{[2]}, \ldots, x_{[p]}$.
\begin{definition}[Holm's procedure]
Let $i^*$ be the largest index such that
$$
\overline{F}(x_{[i]})) \le \alpha / (p-i+1),\quad \text{for all }\;i\le i^*.
$$
Holm's procedure with level $\alpha$ is the thresholding procedure with threshold
\begin{equation} \label{eq:Holm-procedure}
    t_p(x) = x_{[i^*]}.
\end{equation}
\end{definition}
In contrast to the Bonferroni procedure, Holm's procedure is data-dependent.
% It can be shown that Holm's procedure also controls FWER at $\alpha$ level, regardless of dependence in the data.
A closely related, more aggressive (data-dependent) thresholding procedure is Hochberg's procedure \citep{hochberg1988sharper}
%\begin{definition}[Hochberg's procedure]
%Hochberg's procedure 
which replaces the index $i^*$ in Holm's procedure with the largest index such that
$$
\overline{F}(x_{[i]}) \le \alpha / (p-i+1).
$$
%where  $\overline{F}(x)=1-F(x)$ is the survival function.
%\end{definition}

It can be shown that Bonferroni's procedure and Holm's procedure both control FWER at their nominal levels, regardless of dependence in the data.
In contrast, Sid\'ak's procedure and Hochberg's procedure control FWER at nominal levels when data are independent.

Last but not least, we review the famed Benjamini-Hochberg (BH) procedure \cite{benjamini1995controlling}, which aims at controlling FDR.
Recall the order statistics of our observations $x_{[1]} \ge x_{[2]}  \ge \ldots \ge x_{[p]}$.

\begin{definition}[Benjamini-Hochberg's procedure]
Let $i^*$ be the largest index such that
$$
\overline{F}(x_{[i]}) \le \alpha i/p.
$$
The Benjamini-Hochberg procedure with level $\alpha$ is the thresholding procedure with threshold
\begin{equation} \label{eq:BH-procedure}
    t_p(x) = x_{[i^*]},
\end{equation}
\end{definition}
The BH procedure is shown to control the FDR at level $\alpha$ when the statistics are independent.
