

% We review four commonly used procedures in multiple testing. 
% Their optimality properties in the chi-square model \eqref{eq:model-chisq} are established in this Section.

A final ingredient we need, before stating our first main result, is a rate at which the nominal levels of FWER or FDR go to zero.
\begin{definition} \label{def:slowly-vanishing}
We say the nominal level of errors $\alpha = \alpha_p$ vanishes slowly, if
\begin{equation} \label{eq:slowly-vanishing-error}
    \alpha\to 0,\quad \text{and} \quad \alpha p^\delta\to\infty \text{  for any } \delta>0.
\end{equation}
\end{definition}
As an example, the sequence of nominal levels $\alpha_p = 1/\log{(p)}$ is slowly vanishing, while the sequence $\alpha_p = 1/\sqrt{p}$ is not.

\subsection{The exact support recovery problem}
\label{subsec:exact-support-recovery-boundary}

The first main result characterizes the phase-transition phenomenon in the exact support recovery problem under the chi-square model.

\begin{theorem} \label{thm:chi-squared-exact-boundary}
Consider the high-dimensional chi-squared model \eqref{eq:model-chisq} with signal sparsity and size as described in \eqref{eq:signal-sparsity} and \eqref{eq:signal-size}.
The function 
\begin{equation} \label{eq:exact-boundary-chisquared}
    g(\beta) = \left(1 + \sqrt{1-\beta}\right)^2
\end{equation}
characterizes the phase transition of exact support recovery problem.
Specifically, if $\underline{r} > {{g}}(\beta)$, then Bonferroni's, Sid\'ak's, Holm's, and Hochberg's procedures with slowly vanishing (see Definition \ref{def:slowly-vanishing}) nominal FWER levels all achieve asymptotically exact support recovery in the sense of \eqref{eq:support-recovery-success}. 

Conversely, if $\overline{r} < {{g}}(\beta)$, then for any thresholding procedure $\widehat{S}$, we have $\P[\widehat{S}=S]\to0$.
Therefore, in view of Lemma \ref{lemma:risk-exact-recovery-probability}, exact support recovery asymptotically fails for all thresholding procedures in the sense of \eqref{eq:support-recovery-faliure}.
\end{theorem}

The procedures mentioned in Theorem \ref{thm:chi-squared-exact-boundary} are reviewed in Appendix \ref{sec:procedures}. 
Proof of the theorem is found in Appendix \ref{subsec:proof-chi-squared-exact-boundary}. 
% The boundary \eqref{eq:exact-boundary-chisquared} is plotted in Figure \ref{fig:phase-chi-squared}.
Comparisons with parallel results in the Gaussian additive error model \eqref{eq:model-additive} will be drawn in Section \ref{sec:additive-error-model-boundaries}.

\begin{remark} \label{rmk:strong-classification-boundary-2}
Theorem \ref{thm:chi-squared-exact-boundary} predicts that the asymptotic boundaries are the same for all values of the parameter $\nu$.
In simulations (see Section \ref{sec:numerical}), we find this asymptotic prediction to be quite accurate for $\nu\le3$ even in moderate dimensions ($p=100$). 
For $\nu>3$, the phase transitions take place somewhat above the boundary ${g}$.
The behavior is qualitatively similar for the other three phase transitions (see Theorems \ref{thm:chi-squared-exact-approx-boundary}, \ref{thm:chi-squared-approx-boundary}, and \ref{thm:chi-squared-approx-exact-boundary} below).
\end{remark}

\subsection{The exact-approximate support recovery problem}
\label{subsec:exact-approx-support-recovery-boundary}

The next theorem describes the phase transition in the exact-approximate support recovery problem.

\begin{theorem} \label{thm:chi-squared-exact-approx-boundary}
In the context of Theorem \ref{thm:chi-squared-exact-boundary}, 
the function 
\begin{equation} \label{eq:exact-approx-boundary-chisquared}
    \widetilde{g}(\beta) = 1
\end{equation}
characterizes the phase transition of exact-approximate support recovery problem.
Specifically, if $\underline{r} > \widetilde{g}(\beta)$, then the procedures listed in Theorem \ref{thm:chi-squared-exact-boundary} with slowly vanishing nominal FWER levels achieve asymptotically exact-approximate support recovery in the sense of \eqref{eq:support-recovery-success}. 

Conversely, if $\overline{r} < \widetilde{g}(\beta)$, then for any thresholding procedure $\widehat{S}$, the exact-approximate support recovery fails in the sense of \eqref{eq:support-recovery-faliure}.
\end{theorem}

Theorem \ref{thm:chi-squared-exact-approx-boundary} is proved in Appendix \ref{subsec:proof-chi-squared-mix-boundaries}. 

\begin{remark}
The boundary \eqref{eq:exact-approx-boundary-chisquared} was briefly suggested by \citet{arias2017distribution}, who focused exclusively on Gaussian additive error models \eqref{eq:model-additive}.
Unfortunately, it was falsely claimed that the boundary characterized the phase transition of the \emph{exact} support recovery problem, and the alleged proof was left as an ``exercise to the reader''.
This exercise was completed in \cite{gao2018fundamental}, where the correct boundary \eqref{eq:exact-boundary-chisquared} was identified. 

Theorem \ref{thm:chi-squared-exact-approx-boundary} here shows that the boundary \eqref{eq:exact-approx-boundary-chisquared} \emph{does} exist, though for the slightly different \emph{exact-approximate} support recovery problem.
As we will see in Section \ref{sec:additive-error-model-boundaries}, the boundary \eqref{eq:exact-approx-boundary-chisquared} also applies to the exact-approximate support recovery problem in the Gaussian additive error model \eqref{eq:model-additive}.
\end{remark}


\subsection{The approximate support recovery problem}
\label{subsec:approx-support-recovery-boundary}

Our third main result characterizes the phase-transition phenomenon in the approximate support recovery problem in the chi-square model.

\begin{theorem} \label{thm:chi-squared-approx-boundary}
Consider the high-dimensional chi-squared model \eqref{eq:model-chisq} with signal sparsity and size as described in \eqref{eq:signal-sparsity} and \eqref{eq:signal-size}.
The function 
\begin{equation} \label{eq:approx-boundary-chisquared}
    h(\beta) = \beta
\end{equation}
characterizes the phase transition of approximate support recovery problem.
Specifically, if $\underline{r} > {h}(\beta)$, then the Benjamini-Hochberg procedure $\widehat{S}_p$ (defined in Appendix \ref{sec:procedures}) with slowly vanishing (see Definition \ref{def:slowly-vanishing}) nominal FDR levels achieves asymptotically approximate support recovery in the sense of \eqref{eq:support-recovery-success}. 

Conversely, if $\overline{r} < {h}(\beta)$, then approximate support recovery asymptotically fails in the sense of \eqref{eq:support-recovery-faliure} for all thresholding procedures.
\end{theorem}

Theorem \ref{thm:chi-squared-approx-boundary} is proved in Appendix \ref{subsec:proof-chi-squared-mix-boundaries}. 


\subsection{The approximate-exact support recovery problem}
\label{subsec:aprox-exact-support-recovery-boundary}

The last phase transition is in terms of the approximate-exact support recovery risk
\eqref{eq:risk-approx-exact}.

\begin{theorem} \label{thm:chi-squared-approx-exact-boundary}
In the context of Theorem \ref{thm:chi-squared-approx-boundary}, the function 
\begin{equation} \label{eq:approx-exact-boundary-chisquared}
    \widetilde{h}(\beta) = \left(\sqrt{\beta} + \sqrt{1-\beta}\right)^2
\end{equation}
characterizes the phase transition of approximate-exact support recovery problem.
Specifically, if $\underline{r} > \widetilde{h}(\beta)$, then the Benjamini-Hochberg procedure with slowly vanishing nominal FDR levels achieves asymptotically approximate-exact support recovery in the sense of \eqref{eq:support-recovery-success}. 

Conversely, if $\overline{r} < \widetilde{h}(\beta)$, then for any thresholding procedure $\widehat{S}$, the approximate-exact support recovery fails in the sense of \eqref{eq:support-recovery-faliure}.
\end{theorem}

Theorem \ref{thm:chi-squared-approx-exact-boundary} is proved in Appendix \ref{subsec:proof-chi-squared-exact-boundary}. 

\begin{remark}
% This is reflected by the fact that $g(\beta) > \widetilde{g}(\beta) > h(\beta)$, and $g(\beta) > \widetilde{h}(\beta) > h(\beta)$ for all $\beta\in(0,1)$.
% The difficulty of the exact-approximate and approximate-exact problems are a little more difficult to guess
Theorems \ref{thm:chi-squared-exact-boundary} through \ref{thm:chi-squared-approx-exact-boundary} allow us to compare, quantitatively, the required signals sizes in support recovery problems, as well as in the global hypothesis testing problem in the chi-square model \eqref{eq:model-chisq}.
As mentioned in Remark \ref{rmk:global-test-boundary}, there exists a phase transition in the global hypothesis testing problem characterized by the boundary
\begin{equation} \label{eq:detection-boundary-chisquare}
    f(\beta) = 
    \begin{cases}
    \left(1-\sqrt{1-\beta}\right)^2, &\beta>3/4 \\
    \max\{0, \beta-1/2\}, &\beta\le3/4,
    \end{cases}
\end{equation}
which was identified in \citet{donoho2004higher}.
Results in this section indicate that at all sparsity levels $\beta\in(0,1)$, the difficulties of the problems in terms of the required signal sizes have the following ordering
$$
f(\beta) < h(\beta) < \widetilde{g}(\beta) < \widetilde{h}(\beta) < g(\beta),
$$
as previewed in Figure \ref{fig:phase-chi-squared}.
The ordering aligns with our intuition that the required signal sizes increase as we move from detection to support recovery problems.
Similarly, more stringent criteria for error control (e.g., FWER compared to FDR) require larger signals.
We can now also compare $\widetilde{g}(\beta)$ and $\widetilde{h}(\beta)$, whose ordering may not be clear from this line of reasoning.
\end{remark}