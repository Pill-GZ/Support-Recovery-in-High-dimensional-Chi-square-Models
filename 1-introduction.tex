
Multiple testing problems are a staple of modern data-driven studies, where a large number of hypotheses are formulated and screened for their plausibility simultaneously.
This multiplicity of tests brings along a multitude of challenges, and has been a subject of extensive study in recent years.
% in the literature of high-dimensional statistics.

% This paper seeks to establish some new asymptotic optimality results for common multiple testing procedures, including the Bonferroni procedure and the Benjamini-Hochberg procedure, in high-dimensional chi-square models.

In this paper, we study in detail the theoretical limits of multiple problems in high-dimensional additive error models, as well as in an idealized model where the statistics follow independent chi-square distributions,
\begin{equation} \label{eq:model-chisq}
    %x(i) \distras{\mathrm{ind.}} \chi_\nu^2\left(\lambda(i)\right), \quad i=1,\ldots,p.
    x(i) \sim \chi_\nu^2\left(\lambda(i)\right), \quad i=1,\ldots,p,
\end{equation}
where $\chi_\nu^2\left(\lambda(i)\right)$ is a chi-square distributed random variable with $\nu$ degrees of freedom and non-centrality parameter $\lambda(i)$.
We will also look for practical procedures that attain the performance limits, as soon as the problems become theoretically feasible.

% While the additive error models are frequently featured in the literature, the chi-square models need some introduction.

Model \eqref{eq:model-chisq} is motivated in particular by large-scale categorical variable screening problems.
A prototypical application is genome-wide association studies (GWAS), where millions of genetic factors are examined for their potential influence on phenotypic traits.
We briefly introduce the scientific background next, to provide some context.

\subsection{Genome-wide association studies and the chi-square model}
\label{subsec:motivation-chisq}

% introduced through the language of GWAS next.
% These categorical covariate screening problems naturally give rise to the high-dimensional chi-square models, introduced through the language of GWAS next.
% 
% Broadly speaking, GWAS aim to discover genetic variations that are linked to traits or diseases of interested, by testing for associations between the subjects' genetic compositions and their phenotypes.
In a typical GWAS with a case-control design, a total of $n$ subjects are recruited,  consisting of $n_1$ subjects possessing a defined trait, and $n_2$ subjects without the trait serving as controls.
The genetic compositions of the subjects are then examined for variations known as single-nucleotide polymorphisms (SNPs) at an array of $p$ genomic marker locations, and compared between the case and the control group.

Focusing on one specific genomic location, the counts of observed genotypes, if two variants are present, can be tabulated as follows.
% \begin{table}[ht]
% \centering
\begin{center}
    \begin{tabular}{cccc}
    \hline
    & \multicolumn{2}{c}{Genotype} & \\
    \cline{2-3}
    \# Observations & Variant 1 & Variant 2 & Total by phenotype \\
    \hline
    Cases & $O_{11}$ & $O_{12}$ & $n_1$ \\
    Controls & $O_{21}$ & $O_{22}$ & $n_2$ \\
    \hline
    \end{tabular}
    % \caption{Tabulated counts of genotype-phenotype combinations in a genetic association test.}
\end{center}
% \end{table}
Researchers test for associations between the genotypes and phenotypes using, for example, the Pearson chi-square test with statistic
\begin{equation} \label{eq:chisq-statistic}
    x = \sum_{j=1}^2 \sum_{k=1}^2 \frac{(O_{jk} - E_{jk})^2}{E_{jk}},
\end{equation}
where ${E}_{jk} = (O_{j1}+O_{j2})(O_{1k}+O_{2k})/n$.
% are the expected number of observations under the null.
%E_{jk} = \Big(\sum_{l}O_{jl}\Big)\Big(\sum_{l}O_{lk}\Big)\Big/n.

Under the mild assumption that the counts $O_{jk}$'s follow a multinomial distribution (or a product-binomial distribution, if we decide to condition on one of the marginals), the statistic $x$ in \eqref{eq:chisq-statistic} can be shown to have an approximate $\chi^2(\lambda)$ distribution with $\nu=1$ degree of freedom at large sample sizes \citep{agresti2018introduction}. 
Independence between the genotypes and phenotypes would imply a non-centrality parameter $\lambda$ value of zero; if dependence exists, we would have a non-zero $\lambda$ where its value depends on the underlying multinomial probabilities.
More generally, if we have a $J$ phenotypes and $K$ genetic variants, assuming a $J\times K$ multinomial distribution, the statistic will follow approximately a $\chi^2_{\nu}(\lambda)$ distribution with $\nu = (J-1)(K-1)$ degrees of freedom, when sample sizes are large.

The same asymptotic distributional approximations also apply to the likelihood ratio statistic, and many other statistics under slightly different modeling assumptions \cite{gao2019upass}.
These association tests are performed at each of the $p$ SNP marker locations throughout the whole genome, yielding $p$ statistics having approximately (non-)central chi-square distributions, $\chi_{\nu(i)}^2\left(\lambda(i)\right)$, for $i=1,\ldots,p$,
% \begin{equation} \label{eq:model-chisquare-approx}
%     x(i) \mathrel{\dot\sim} \chi_{\nu(i)}^2\left(\lambda(i)\right), \quad i=1,\ldots,p,
% \end{equation}
where $\lambda = (\lambda(i))_{i=1}^p$ is the $p$-dimensional non-centrality parameter.
%, with $\lambda(i)=0$ indicating independence of the $i$-th SNP with the outcomes, and $\lambda(i)\neq0$ indicating associations.

Although the number of tested genomic locations $p$ can sometimes exceed $10^5$ or even $10^6$, it is often believed that only a small set of genetic locations have tangible influences on the outcome of the disease or the trait of interest.
Under the stylized assumption of sparsity, $\lambda$ is assumed to have $s$ non-zero components, with $s$ being much smaller than the problem dimension $p$. 
The goal of researchers is two-fold: 1) to test if $\lambda(i)=0$ for all $i$, and 2) to estimate the set $S=\{i:\lambda(i)\neq 0\}$.
In other words, we look to first determine if there are \emph{any} genetic variations associated with the disease; and if there are associations, we want to locate them.
The latter, referred to as the \emph{support recovery problem}, is the focus of this work.

\subsection{One-sided vs two-sided alternatives in additive error models}
\label{subsec:motivation-additive}

The chi-square model \eqref{eq:model-chisq} plays an important role in analyzing variable screening problems under omnidirectional alternatives.
A primary example is multiple testing under two-sided alternatives in the additive error model,
\begin{equation} \label{eq:model-additive}
    x(i) = \mu(i) + \epsilon(i), \quad i=1,\ldots,p,
\end{equation}
where the errors $\epsilon$ are assumed to have standard normal distributions.

Under two-sided alternatives, unbiased test procedures call for rejecting the hypothesis $\mu(i)=0$ at locations where observations have large absolute values, or equivalently, large squared values.
Squaring on both sides of \eqref{eq:model-additive}, we arrive at Model \eqref{eq:model-chisq} with non-centrality parameters $\lambda(i) = \mu^2(i)$ and degree-of-freedom parameter $\nu =1$.
In this case, the support recovery problem is equivalent to locating the set of observations with mean shifts, $S=\{i:\mu(i)\neq 0\}$, where the mean shifts could take place in both directions.

Therefore, a theory for the chi-square model \eqref{eq:model-chisq} naturally lends itself to the study of two-sided alternatives in the Gaussian additive error model \eqref{eq:model-additive}.
In comparing such results with existing theory on one-sided alternatives \cite{arias2017distribution, gao2018fundamental}, we will be able to quantify if, and how much of a price has to be paid for the additional uncertainty when we have no prior knowledge on the direction of the signals.

% In many applications, of course, restricting ourselves to one-sided tests is unrealistic.
% For example, in fMRI studies, the interest is in \emph{both} regions where average brain activities {increase} and where they {decrease}, when comparing the case group to the controls \citep{narayan2015two}. 
% In the challenging application of anomaly detection on Internet traffic streams, millions of IP addresses need to be scanned in real time to identify \emph{both} volumetric attacks and blackouts \citep{kallitsis2016amon}.
% Indeed, omnidirectional tests are the more natural choice in so-called discovery sciences where little to no prior knowledge is available.

\subsection{Asymmetric statistical risks}
\label{subsec:asymmetric-risk}

% An issue of practical importance that was somewhat overlooked by the literature is the choice of statistical risks.
The third consideration that motivates us is the choice of practically relevant statistical risks.
To the best of our knowledge, so far, the literature on multiple testing have been concerned with criteria roughly \emph{symmetric} in terms of false discovery and false non-discovery control.
For example, \citet{arias2017distribution} studied conditions under which \emph{fractions} of false discovery and \emph{fractions} of non-discovery can vanish; \citet{gao2018fundamental} investigated \emph{family-wise error control} for both types of errors; meanwhile, \citet{butucea2018variable} analyzed the problem under the Hamming loss, which penalizes false discoveries and missed signals {equally}.

In applications, however, attitudes towards type I and type II errors are often different.
In the example of GWAS, where the number of candidate locations $p$ could be in the millions, researchers are typically interested in the marginal (location-wise) power of discovery, while exercising stringent (family-wise) false discovery control --- a situation not covered by existing theory.
The observation on this asymmetry leads us to study, and discover, two new phase transitions, both in the additive error model \eqref{eq:model-additive} under one-sided alternatives, and in the chi-square model \eqref{eq:model-chisq}.
The latter, as discussed in Section \ref{subsec:motivation-additive}, entails the additive error model \eqref{eq:model-additive} under two-sided alternatives.
% Further details can be found in Section \ref{subsec:risks} below, after 
% We summarize the main messages of this paper next.

\subsection{Contributions}

Our contribution is three-fold: (1) a thorough study of the theoretical limits of thresholding procedures in the chi-square model \eqref{eq:model-chisq} and an enumeration of practical procedures that attain these limits, 
(2) an illustration of the implications of the phase transition phenomena in genetic association studies, and
(3) two new optimality results in the Gaussian additive error model \eqref{eq:model-additive}, as well as direct comparisons of the one-sided and two-sided alternatives. 

In detail, we show that several commonly used family-wise error rate-control procedures --- including Bonferroni's procedure \cite{dunn1961multiple} --- are asymptotically optimal for the \emph{exact}, and \emph{exact-approximate} support recovery problems (defined in Section \ref{subsec:risks} below) in the chi-square model \eqref{eq:model-chisq}.
We further show that the Benjamini-Hochberg (BH) procedure \cite{benjamini1995controlling} is asymptotically optimal for the \emph{approximate}, and \emph{approximate-exact} support recovery problems (defined, too, in Section \ref{subsec:risks}).
These four results are made precise in our main theorems in Section \ref{sec:chisq-boundaries}.
% \ref{thm:chi-squared-exact-boundary}, \ref{thm:chi-squared-exact-approx-boundary}, \ref{thm:chi-squared-approx-boundary}, and \ref{thm:chi-squared-approx-exact-boundary}.
Under appropriate parametrizations of the signal sizes and sparsity, they establish the phase transitions of support recovery problems in the chi-square model, previewed in Figure \ref{fig:phase-chi-squared}.
Remarkably, the degree-of-freedom parameter does not affect the asymptotic boundaries in any of the four support recovery problems.

We then return to association screenings of categorical variables in Section \ref{sec:signal-size-odds-ratio}, and 
present the consequences of the phase transition in the exact-approximate problem in large-scale genetic association studies.
% present empirical evidence for the phase transition in the exact-approximate problem using real data from large-scale association studies on breast cancer obtained from the NHGRI-EBI GWAS Catalog \cite{macarthur2016new}.
% demystify the notion of signal size $\lambda$ in this context.
% which is perhaps less transparent than in additive error models.
We do so by characterizing the relationship between the signal size $\lambda$ and the marginal frequencies, odds ratio, and sample sizes for association tests on 2-by-2 contingency tables.
% Specifically, the amount of signal, when rare variants are present, is weaker compared to the signal when the marginal distributions are balanced.
% In other words, reliable detection of the effects by rare variants would require more samples compared to common variants, even at the same odds ratio.
This result, establishing the relationship between sample sizes and signal sizes, is made precise in Proposition \ref{prop:signal-size-odds-ratio}.
A corollary of the proposition, which may be of independent interest, enables us to determine the optimal design for association studies under a fixed budget, and reveals that balanced designs with equal number of cases and controls are often statistically inefficient.
We illustrate practical use of these results in power analysis with numerical examples. 

Finally, we study the Gaussian additive error model \eqref{eq:model-additive} under \emph{one-sided alternatives} in Section \ref{sec:additive-error-model-boundaries}. 
% While phase transitions for the approximate and exact support recovery problems were discovered in \cite{arias2017distribution} and \cite{gao2018fundamental}, 
Two new results on the phase transition of support recovery problems are established.
All phase transition boundaries (under suitable parametrizations) coincide with those in the chi-square models.
% and Figure \ref{fig:phase-chi-squared} continues to apply.
Following the discussion in Section \ref{subsec:motivation-additive}, this indicates vanishing differences between the difficulties of the one-sided and two-sided alternatives in the Gaussian additive error model \eqref{eq:model-additive}.

The rest of this paper is organized as follows. 
Section \ref{sec:setup} describes the formal set-up for the main results summarized above; these main results are detailed in Sections \ref{sec:chisq-boundaries}, \ref{sec:signal-size-odds-ratio}, and \ref{sec:additive-error-model-boundaries}. 
The phase transitions are demonstrated with numerical simulations in Section \ref{sec:numerical}.
%The many interesting results obtained under one-sided alternatives cannot be easily adapted to the two-sided cases where the locations and the directions of the signals are both unknown.
More related literature, comments, and discussions are collected in Section \ref{sec:discussions}.
Appendix \ref{sec:procedures} contains brief review of commonly used procedures in multiple testing.
Proofs are presented in Appendix \ref{sec:proofs}.


% Practical issues are also addressed to make for simple and effective power analysis.
